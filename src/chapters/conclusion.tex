\chapter{Conclusion}\label{cha:conclusion}

The primary objective of this research was to explore the feasibility of differentiating SSVEP responses based on the eye being visually stimulated. Additionally, the study aimed to investigate the possibility of eliciting separate SSVEP responses in a VR environment. These objectives were set within the broader context of advancing BCI technologies.

The study tackled multiple research questions across different experiments, each contributing to a nuanced understanding of SSVEP responses and their underlying neural mechanisms. Experiment 1 primarily focused on the feasibility of differentiating hemisphere-specific SSVEP responses based on the eye being visually stimulated. While central channels like POz and Oz largely upheld the null hypothesis, suggesting they are not ideal for this differentiation, lateral channels such as PO3, O1, and O2 provided evidence against the null hypothesis, particularly when the left eye was compared to both eyes or the right eye. This lends partial support to the feasibility of differentiating SSVEP responses based on the eye being stimulated. Experiment 2 took this a step further by exploring the ability to elicit separate SSVEP responses in a Virtual Reality (VR) environment using unique frequency stimuli for each eye. The evidence strongly supported this ability. Finally, Experiment 4 added complexity to the picture by investigating the role of directed attention in binocular rivalry. The findings were mixed, suggesting that the role of attention in this phenomenon is influenced by multiple factors, including the limitations of the EEG device used.

The findings suggest that contralateral processing theories hold true for lateral channels, but the central channels' location makes them less reliable for this differentiation. The evidence from Experiment 2 supports the concept of dual-SSVEP in VR, opening up new avenues for BCI applications. However, the complexities observed in Experiment 4 indicate that directed attention and binocular rivalry are influenced by a myriad of factors, including the limitations of the EEG device used.

The study had several limitations, including the spatial resolution of the EEG device, which affected the reliability of the findings. Anomalous results in channels like PO4 also point to potential variables like eye dominance that were not accounted for. Additionally, the study employed non-parametric tests due to the non-Gaussian distribution of the data, which may not be the most robust method for all conditions.

Given the limitations and the complexities observed, future research should consider using more advanced neuroimaging techniques to delve deeper into the neural processes underlying SSVEP responses and binocular rivalry. Studies could also explore the influence of peripheral vision stimuli on SSVEP responses, especially given the contradictory findings observed in the amplitude values during the first condition and the inconsistencies between PO3 and PO4 channels. Further work could also investigate the role of eye dominance in SSVEP responses, which could be a significant variable affecting the results.

In summary, while the study provides valuable insights into the differentiation of SSVEP responses and the feasibility of dual-SSVEP in VR, it also raises several questions that warrant further investigation. The findings have important implications for the development of more sophisticated and reliable BCIs but also underscore the need for continued research in this rapidly evolving field.
