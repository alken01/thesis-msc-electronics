\chapter{Discussion}\label{cha:discussion}

\section{Discussion}

\subsection{Differentiation of Hemisphere-Specific SSVEP Responses Based on the Visually Stimulated Eye}

The research question probed the feasibility of differentiating hemisphere-specific SSVEP responses contingent on the eye subjected to visual stimulation. The null hypothesis posited that no such differentiation exists.

\subsubsection{Methodological Considerations}

To evaluate the observed differences, a significance level of \( \alpha = 0.05 \) was employed. Given the non-Gaussian distribution of the PSD, non-parametric tests were favored over traditional parametric tests. Specifically, the Wilcoxon signed-rank test was utilized, which is particularly apt for data that deviate from a normal distribution.

\subsubsection{Electrode-Specific Findings}

\paragraph{Central Channels: POz and Oz}
In alignment with expectations, the null hypothesis was predominantly upheld for the central channels, with the exception of the Oz channel under the third condition. This outcome is not entirely surprising given the central location of the POz and Oz electrodes, which are positioned in a region close to both hemispheres. As such, they are likely to capture a blend of neural activity from both sides, potentially diluting the hemisphere-specific responses that might have been observed with more laterally placed electrodes.

\paragraph{Lateral Channels: PO3, O1, and O2}
The lateral channels, particularly PO3 and O1, showed distinct patterns across the three conditions. In the first condition, comparing the left eye being stimulated versus both eyes, and in the second condition, comparing the left eye versus the right eye, both channels provided strong evidence against the null hypothesis. These findings are consistent with contralateral processing theory, as these channels are situated in the left hemisphere and primarily process input from the right eye.

In the third condition, where both eyes were compared to the right eye, the null hypothesis was not rejected for the PO3 and O1 channels. This outcome is primarily a reflection of the EEG device's limited spatial resolution. While these channels are located in the left hemisphere and process input from the right eye, the device's limitations make it challenging to distinguish between the right eye being stimulated and both eyes being stimulated.

For the O2 channel, the null hypothesis was not rejected when comparing the left eye being stimulated versus both eyes. This outcome is primarily attributed to the limited spatial resolution of the EEG device. The O2 channel, situated in the right hemisphere, processes input from the left eye. However, the device's limitations make it challenging to distinguish between the left eye being stimulated and both eyes being stimulated.

\paragraph{Anomalous Findings: PO4 Channel}
The PO4 channel presented an unexpected pattern, especially when compared to its counterpart, PO3. This discrepancy could be attributed to multiple factors, including the limited spatial resolution of the EEG device and the potential influence of eye dominance, a less explored but potentially significant variable.

\subsubsection{Population PSD Analysis}
Population Power Spectral Density (PSD) was analyzed across three regions: left (PO3, O1), middle (POz, Oz), and right (PO4, O2), as depicted in Figure~\ref{fig:e1_population}. The amplitude values across different conditions offered varying degrees of support for the research question. Some conditions contradicted established theories of contralateral processing, while others reinforced them.

\paragraph{First Condition: Contradictory Findings}
The amplitude values in the first condition contradicted both the research question and the established understanding of contralateral processing. A speculative explanation could be the influence of peripheral vision stimuli on SSVEP responses, although this hypothesis currently lacks empirical support.

\paragraph{Second Condition: Dominant Eye Effect}
The second condition revealed amplitude values that align with the notion of a dominant right eye, as evidenced by higher amplitude in the left hemisphere. This finding further substantiates the theory of contralateral processing.

\paragraph{Third Condition: Support for Research Question}
The third condition provided evidence in favor of the research question, demonstrating a higher amplitude in the left hemisphere when the right eye was stimulated. The elevated amplitude in the middle region could be attributed to a combination of eye dominance and the limitations of the EEG device.



\section{Experiment 2}

\subsection{Research Question 1: Differentiation of SSVEP Responses Based on Unique Frequency Stimuli}

The first research question aimed to differentiate SSVEP responses based on the unique frequency stimuli presented to each eye. The p-values from Tables \ref{tab:ex2-66}, \ref{tab:ex2-75}, and \ref{tab:ex2-ratios} already indicated a nuanced ability to differentiate these responses, especially in the left and middle regions. This evidence is further supported by the Population PSD Figure \ref{fig:e2_population}, which shows distinct amplitude peaks for both stimulating frequencies, 6.6 Hz and 7.5 Hz.

\subsection{Research Question 2: Simultaneous Elicitation of Separate SSVEP Responses}

\paragraph{First Trial}
Both unique frequencies (6.6 Hz and 7.5 Hz) elicited distinct SSVEP responses in both hemispheres. The p-values had already indicated that the amplitude for the 7.5 Hz frequency was higher in both hemispheres, contradicting the dominant eye theory. The absence of IM components highlights the advantages of using VR to control stimuli for each eye separately. The Population PSD Figure \ref{fig:e2_population} further supports this.

\paragraph{Second Trial}
The p-values showed a clear peak at 6.6 Hz across all regions, reinforcing the feasibility of a dual-SSVEP paradigm in VR. This is further supported by the Population PSD Figure \ref{fig:e2_population}.

\paragraph{Third Trial}
The p-values indicated that the amplitude of the 6.6 Hz frequency was higher than that of the 7.5 Hz. This is further supported by the Population PSD Figure \ref{fig:e2_population}, and the reason for this phenomenon remains unclear but warrants further investigation.

\paragraph{Amplitude Observations}
The Population PSD Figure \ref{fig:e2_population} shows distinct amplitude peaks for both stimulating frequencies, 7.5 Hz and 6.6 Hz, further substantiating the p-value evidence that dual-SSVEP has been achieved in a virtual reality environment.

\paragraph{Presence of Dual-SSVEP}
The p-values and the presence of well-defined amplitude peaks at both stimulating frequencies substantiate the concept of achieving dual-SSVEP responses within the immersive domain of virtual reality.
.

\section{Experiment 4}

\subsubsection{Individual Frequency Comparisons}

\paragraph{7.5 Hz Frequency:} 
The comparison between Experiment 3 and the conditions of Experiment 4 for the 7.5 Hz frequency yielded mixed results. While most channels did not provide sufficient evidence to reject the null hypothesis, channels PO3, Oz, and PO4 showed evidence against it in the comparison with the second condition of Experiment 4.

\paragraph{6.6 Hz Frequency:} 
For the 6.6 Hz frequency, the channels largely supported the null hypothesis, except for PO3, POz, and Oz when comparing Experiment 3 with the second condition of Experiment 4.

\subsubsection{Ratio of Amplitude Frequencies}

The comparison of the amplitude ratios between the 6.6 Hz and 7.5 Hz frequencies predominantly supported the null hypothesis, except for specific channels like PO3 and Oz.

\subsubsection{Overall Interpretation and Practical Implications}

The individual frequency comparisons and the ratio comparisons collectively suggest that the role of directed attention in binocular rivalry is complex and not easily influenced by conscious control. The inconsistencies among channels could either indicate that these channels are more sensitive to changes in attentional focus or reflect the limitations of the EEG device used. These findings underscore the need for further research using more advanced neuroimaging techniques to delve into the complexities of neural processes underlying binocular rivalry.

